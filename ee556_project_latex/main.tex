\documentclass[a4paper]{article} 
\input{head}
\begin{document}

%-------------------------------
%	TITLE SECTION
%-------------------------------

\fancyhead[C]{}
\hrule \medskip % Upper rule
\begin{minipage}{0.295\textwidth} 
\raggedright
\footnotesize

\end{minipage}
\begin{minipage}{0.4\textwidth} 
\centering 
\large 
EE 556 Project\\ 
\normalsize 
\end{minipage}
\begin{minipage}{0.295\textwidth} 
\raggedleft
\today\hfill\\
\end{minipage}
\medskip\hrule 
\bigskip

%-------------------------------
%	CONTENTS
%-------------------------------

\section{Code}

Please download the folder `556project' from canvas. The code is implemented in Matlab.

\subsection{Requirements}
Matlab 2020b (Other version may also work)

Deep Learning toolbox installed in Matlab.




\subsection{Description}

There are 4 files in the `556project' folder. `data.mat' stores the MNIST training and test data, `net.mat' stores the data and trained network. Using `network.m', we can load the data and train a CNN. `test.m' provides the code for getting network outputs, showing images, calculating accuracy.

\subsection{MNIST dataset}

In this project, the MNIST dataset is used, MNIST is a hand write digit dataset. There are 60000 images in the training set and 10000 images in the test set. The sizes of all the images are 28 * 28.
the dataset is available in the following link:

http://yann.lecun.com/exdb/mnist/

However, we downloaded the dataset and processed it, so that we can easily use in Matlab. The processed dataset is stored in `data.mat'.


4 variables are stored in `data.mat': `trainX'(training images), `trainY'(training labels), `testX'(test images), `testY'(test labels).  

\subsection{Question}
If you have any question about the code, please send an email to hzw81@psu.edu


\end{document}
